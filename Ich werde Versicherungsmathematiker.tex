\documentclass[12pt]{report}

\usepackage[a4paper,width=150mm,top=25mm,bottom=25mm,bindingoffset=6mm]{geometry}
\usepackage[onehalfspacing]{setspace}
\usepackage{ucs}
\usepackage[table,xcdraw]{xcolor}
\definecolor{mColor1}{rgb}{0.9,0.9,0.9}

\usepackage{fancyhdr}
\pagestyle{fancy}
\fancyhead{}
\renewcommand{\chaptermark}[1]{\markboth{#1}{}}
\renewcommand\sectionmark[1]{\markright{\thesection\ #1}}

\fancyhead[LO, RE]{\leftmark}
\fancyhead[LE, RO]{\rightmark}

\usepackage{titlesec, blindtext, color}
\definecolor{gray75}{gray}{0.75}
\usepackage{mathptmx}
\usepackage[utf8]{inputenc}
\usepackage[T1]{fontenc}
\usepackage[ngerman]{babel}

\usepackage{amsmath,amssymb,amstext,amsthm,mathtools}
\usepackage{url}
\usepackage{caption}
%\usepackage[belowskip=-5pt,aboveskip=0pt]{caption}
\usepackage{subcaption}

\usepackage{float}
\usepackage{lscape}
\usepackage{pdfpages}
\usepackage{rotating}
\usepackage{graphicx}
\setlength\parindent{0pt}
\usepackage{hyperref}
\usepackage{acronym}
\usepackage{textcmds}
\usepackage{longtable}
\usepackage[export]{adjustbox}
\usepackage{upgreek}
\usepackage{dsfont}
\usepackage{tensor}


\DeclareMathAlphabet{\mathcal}{OMS}{cmsy}{m}{n}
\SetMathAlphabet{\mathcal}{bold}{OMS}{cmsy}{b}{n}

\usepackage{listings, lstautogobble}
\usepackage{textcomp}
\definecolor{yo}{rgb}{0.9,0.6,0}
\definecolor{Gray}{gray}{0.9}
\definecolor{listinggray}{gray}{0.9}
\definecolor{lbcolor}{rgb}{0.95,0.95,0.95}
\lstset{
	backgroundcolor=\color{lbcolor},
	tabsize=4,
	rulecolor=,
	language=python,
        basicstyle=\scriptsize,
        upquote=true,
        aboveskip={1.5\baselineskip},
        columns=fixed,
        showstringspaces=false,
        extendedchars=true,
        breaklines=true,
        prebreak = \raisebox{0ex}[0ex][0ex]{\ensuremath{\hookleftarrow}},
        frame=lines,
        showtabs=false,
        showspaces=false,
        showstringspaces=false,
        identifierstyle=\ttfamily,
        keywordstyle=\color[rgb]{0.55,0,0},
        alsoletter={/,*,[,]},%
        otherkeywords={},
        morekeywords=[2]{with, as},
        morekeywords=[3]{},
        emph={self},          % Custom highlighting
		emphstyle=\color[rgb]{0.1,0.3,1},
		emph={[2]f},          % Custom highlighting
		emphstyle={[2]\color[rgb]{0.1,0.5,0.1}},
		emph={[3]__init__},          % Custom highlighting
		emphstyle={[3]\color[rgb]{0.1,0.3,1}},
		emph={[4]open,str,print,KeyError},          % Custom highlighting
		emphstyle={[4]\color[rgb]{0.2,0.6,0.8}},
        commentstyle=\color[rgb]{0.3,0.3,0.3},
        stringstyle=\color[rgb]{0.133,0.545,0.133},
        	autogobble=true
}
\lstnewenvironment{ttlisting}{\lstset{basicstyle=\scriptsize}}{}

\usepackage{color}
\usepackage[section]{placeins}

\newenvironment{simplechar}{%
	\catcode`\$=12
	\catcode`\&=12
	\catcode`\#=12
	\catcode`\^=12
	\catcode`\_=12
	\catcode`\~=12
	\catcode`\%=12
	\catcode`\"=12
	\catcode`\'=12
	}{}{}

\newtheoremstyle{dotless}{}{}{\itshape}{}{\bfseries}{}{ }{}

\theoremstyle{dotless}

\newtheorem{thm}{Theorem}
\newtheorem{defn}[thm]{Definition} 
\newtheorem{exmp}[thm]{Example}
\theoremstyle{definition}


\begin{document}

\begin{titlepage}
	Warum bin ich nicht einfach Staubsaugervertreter geworden? 
\end{titlepage}

\tableofcontents

\chapter{Allgemeiner Unsinn für Grundlagen aktuarieller Kalkulation}

\subsubsection{Sparten}
\begin{itemize}
	\item Umfasst Leben, Kranken, Komposit, Pensionen
	\item Leben, Kranken, Pensionen sind zusammen Personenversicherung
	\item Komposit: Schaden/Unfall
	\item Besonders: priv. Unfall ist Komposit
\end{itemize}

\begin{defn} (Farny)
	Deckung eines im Einzelnen ungewissen, insgesamt schätzbaren Mittelbedarfs unter Nutzung von Ausgleichsmechanismen im Kollektiv.
\end{defn}

\subsubsection{Wichtigste Zweige Komposit}
\begin{itemize}
	\item Sachversicherung
	\item Haftpflichtversicherung
	\item Transportversicherung
	\item Technische Versicherung
\end{itemize}

\subsubsection{Prämienzahlweise}
\begin{itemize}
	\item üblicherweise jährlich
	\item bei unterjährigen Zahlung Ableitung aus Jahresprämie
\end{itemize}

\subsubsection{Diskont und Barwert}
\begin{itemize}
	\item Diskontfunktion bei einjährigem Zinssatz $r$: $D(t)=(1+r)^{-t}$
	\item Diskontfunktion bei Rechnungszins $i$: $D(t) = (\frac{1}{1+i})^t \eqqcolon v^t$
	\item Barwert aller Leistungen: $L=\sum_{t=0}^{\bar{n}}D(t)\cdot L_t$
	\item Barwert aller Prämien: $P=\sum_{t=0}^{\bar{n}}D(t)\cdot P_t$
	\item Barwert aller Kosten: $K=\sum_{t=0}^{\bar{n}}D(t)\cdot K_t$
\end{itemize}

\subsubsection{Äquivalenzprinzip}
\begin{align}
	\text{(ÄP I):} \ \ \ \ \ \ \ \ \ \ \ \ \ \ \ E(P)=E(L) \\
	\text{(ÄP II):} \ \ E(P)=E(L)+E(K)
\end{align} 

\begin{defn} \ \\
	\begin{itemize}
		\item Falls $L$ und $P$ das Äquivalenzprinzip erfüllen, dann hei{\ss}t $P_{\bullet}$ Nettorisikoprämienprozess und $P_t$ Nettorisikoprämie.
		\item $L$ und $P$ erfüllen ÄP und $\exists \ w_t$ Wahrscheinlichkeit der Prämienzahlung $P_t$ und $\bar{P}$ konstant mit $E(P_t)=\bar{P}\cdot w_t \ \forall \ t \in \{0,...,\bar{n}\}$. $\bar{P}$ konstante Nettorisikoprämie.
		\item Bruttorisikoprämie: $P^+ \coloneqq \bar{P}+c$ mit $c>0$ Sicherheitszuschlag.
		\item Alternativ: Sicherheitszuschlag bereits in Nettorisikoprämie enthalten
	\end{itemize}
\end{defn}

\subsubsection{Notation}
\begin{itemize}
	\item $\bar{n}$: Modelldauer
	\item $t$: Zeit in jahren
	\item $r$: einjähriger konstanter Zinssatz
	\item $D(t)$: Diskontfunktion
	\item $L_t$: Versicherungsleistung in $t$
	\item $q_t$: Eintrittswahrscheinlichkeit Leistungsfall in $t$
	\item $P_t$: Prämienzahlung in t
	\item $w_t$: Wahrscheinlichkeit Prämienzahöung in $t$
	\item $K_t$: Kosten in $t$
	\item $L$: Leistungsbarwert
	\item $P$: Prämienbarwert
	\item $K$: Kostenbarwert
\end{itemize}

\subsubsection{Sterbetabeln}

\begin{figure}[ht]
	\centering
	\includegraphics[width = .8\textwidth]{Bilder/Sterbetafelbsp.png}
\end{figure}

\subsubsection{Allgemeine aktuarielle Herangehensweise, spartenübergreifend ähnliches Standardvorgehen zur Bewertung zufälliger zukünftiger Versicherungsleistungen}
\begin{itemize}
	\item Beobachtung von Vergangenheit (Daten) zur Vorhersage der Zukunft
	\item Anpassung geeigneter Wahrscheinlichkeitsverteilung
	\item Sorgfalt bzgl. möglicher Änderungen von Annahmen im zeitlichen Verlauf
	\item typischerweise konstante Prämienhöhe
	\item Risiko steigt mit zeitlichem Verlauf
	\item Ansparprozess und Entsparprozess
\end{itemize}

\subsubsection{Rückstellungen}
\begin{itemize}
	\item Ziel: Sicherstellung der dauernden Erfüllbarkeit
	\item versicherungstechnische Rückstellungen wichtigste Passivposition in der Bilanz des VU
	\item hohe bedeutung für interne Unternehmensbewertung
	\item Einfluss auf Besteuerung des VU
	\item Unterschied zwischen bilanzieller und einzelvertraglicher versicherungsmathematischer Deckungsrückstellung
	\item Deckungskapitel $\hat{=}$ Erwarteter Barwert künftiger Leistungen - Erwarteter Barwert künftiger Beiträge
\end{itemize}

\subsubsection{Rückstellungen in der Schadenversicherung}
\begin{itemize}
	\item Einzelschadenreserven: für noch nicht vollständig abgewickelte Schäden
	\item Deckungsrückstellungen: für Haftpflicht, Unfallrenten und Beitragsrückgewähr in Unfall
	\item Spätschadenpauschalreserve: für IBNR
	\item Schwankungsrückstellung: relevant für Zweige mit stark variierenden Schadenfällen 
\end{itemize}

\subsubsection{Prämienprinzipien}
\begin{itemize}
	\item Ziel: Zuordnung angemessener Prämie durch Bemessung geeigneter Sicherheitszuschläge
	\item Deckung der Leistungsfälle und zusätzliche Prämie zur Bereitschaft der Risikoübernahme durch VU (Sicherheitszuschlag SZ(X))
	\item Prämienprinzipien $H(X) \coloneqq E(X)+SZ(X)=P^+$, $X$ ist das versicherte Risiko
	\item Sicherheitszuschlag bei gleichem EW höher, wenn Risiko gefährlicher
	\item Nettorisikoprinzip: $H(X)=E(X)$
	\item Erwartungswertprinzip: $H(X)=E(X)+\delta \cdot E(X) = (1+\delta) \cdot E(X)$
	\item Varianzprinzip: $H(X)=E(X) + \delta \cdot Var(X)$
	\item Standardabweichungsprinzip: $H(X)=E(X) + \delta \cdot \sqrt{Var(X)} = E(X) + \delta \cdot \sigma(X)$
	\item Exponentialprinzip: $H(X) = \frac{1}{a} \cdot ln(M_X(a)) = \frac{1}{a}\cdot ln(E[e^{aX}])$ mit $a>0$, Monumenterzeugender Funktion $M_X$, entspricht näherungsweise Varianzprinzip mit $\delta = \frac{a}{2}$
\end{itemize}

\begin{defn}
	(Ungleichung von Centelli)
	$P(X>E(X)+c) \leq \frac{Var(X)}{c^2+Var(X)}$ \\
	Hinweis: SZ wird hier stark überschätzt.
\end{defn}

\subsubsection{Beispiele Risikoma{\ss}e}
\begin{itemize}
	\item Erwartungswert $E(X)$
	\item Varianz $Var(X)$
	\item Schiefe $\gamma(X)$ (Symmetriema{\ss}
	\item Tail-Whk $P(X>t)$
	\item Ruin- und Verlustwahrscheinlichkeiten
	\item Bernoulli-Nutzen
	\item Value at Risk (VaR), Expected Shortfall, Tail Value at Risk (TVaR)
\end{itemize}

\begin{defn}
\begin{align}
	\text{Additivität: } H(X+Y)=H(X)+H(Y) \ \forall \ X,Y \text{stochastisch unabhängig} \\
	\text{Subadditivität: } H(X+Y) \leq H(X)+H(Y) \ \forall \ X,Y \text{stochastisch unabhängig} \\
	\text{Erwartungswertübersteigend: } SZ(X) \geq 0
\end{align}
\end{defn}

% To-Do: Folie 112-114 beantworten und einfügen

\begin{defn}
	(1) Ein Kollektiv stellt eine Zusammenfassung von Risiken dar, die durch gleichartige Gefahren bedrohnt sind. Kollektiv bedeutet nicht zwangsläufig, dass es sich um \textit{versicherte Risiken} handelt. \\
	(2) Der Risikoausgleich im Kollektiv stellt neben dem Ausgleich in der Zeit ein wesentliches Funktionsprinzip von Versicherungen dar. \\
	(3) Ein Kollektiv hei{\ss}t homogen, falls alle Risiken des Kollektivs dieselbe Verteilung besitzen, anderenfalls hei{\ss}t es heterogen. \\
	(!) Hinweis: Homogenität und Unabhängigkeit sind keine notwendige Voraussetzung für Risikoausgleich im Kollektiv. Im Gegenteil: gleicht sich durch gegenläufige Abhängigkeiten z.T. aus.
\end{defn}

\subsubsection{Risikoausgleich}
\begin{itemize}
	\item Das Überschreiten einer prozentualen Maximalabweichung vom Erwartungswert wird bei wachsendem Kollektiv immer unwahrscheinlicher.
	\item Risikoausgleich im Kollektiv erfolgt insofern, als dass der Variationskoeffizient als versicherungsspezifisches Risikoma{\ss} für wachsende Bestände gegen 0 konvergiert.
	\item Mit zunehmender Zahl von Risiken sinkt die relative Abweichung des arithmetischen Mittels vom Erwartungswert.
\end{itemize}

\begin{defn}
$Y_i\geq$ kumulierter Gesamtaufwand des $i$-ten Risikos. $S^{ind}=\sum_{i=1}^nY_i $.
	\begin{align}
		\text{Durch Linearität des EWs: } E(S^{ind})=\sum_{i=1}^n E(Y_i) \\
		\text{Da } Y_i \text{ unabhängig: } Var(S^{ind})=\sum_{i=1}^n Var(Y_i) \\
		\text{Variationskoeffizient: } Vko(S^{ind})=\frac{\sqrt{\sum_{i=1}^nVar(Y_i)}}{\sum_{i=1}^nE(Y_i)}
	\end{align}
\end{defn}

\begin{defn}
	Erste und zweite Formel von Wald. N die Schadenzahl. \\
	(1) $E(S^{koll})=E(N)\cdot E(X)$ \\
	(2) $Var(S^{koll})=E(N) \cdot Var(X) + (E(X))^2 \cdot Var(N)$
\end{defn}

\subsubsection{Gegenüberstellung individuelles und kollektives Modell}
\begin{itemize}
	\item dieselbe Gesamtsumme $S^{int}=S^{koll}$
	\item im individuellen Modell Aggregation der einzelnen Aufwände pro Risiko und Zeitraum erforderlich
	\item kollektives Modell: Betrachtung einzelner Ereignisse ohne Erfassung, welches Risiko den Aufwand verursacht
	\item i.A. bietet das KM eine bessere Basis für die Schätzung der Verteilung
	\item Annahme identisch verteilter Aufwände bei IM nur näherungsweise erfüllt 
\end{itemize}


\subsubsection{Zustandsmodell der Personenversicherung}

\begin{figure}[ht]
	\centering
	\includegraphics[width = .8\textwidth]{Bilder/ZustandsmodellPersVers}
\end{figure}
 \begin{itemize}
 	\item Modellannahmen nicht immer sachgerecht
 	\item Markov-Eigenschaft kritisch: Relevant, ob \textit{aktiv} $\rightarrow$ \textit{Rente} oder \textit{invalide} $\rightarrow$ \textit{Rente}
 	\item z.T. sehr viele Zustände erforderlich (z.B. Abhängigkeit der Leistungshöhe von Anzahl Dienstjahren, bei Invalidität der Zeitpunkt des Eintritts in den Invalidenstatus)
 \end{itemize}

\subsubsection{Risikoteilung}

\begin{itemize}
	\item teilweiser Risikotransfer im direkten Geschäft zwischen VN und Erstversicherer sowie im Rahmen von Rückversicherung (RV)
	\item Risikoteilung im Direktgeschäft: Selbstbehalt beim VN, genannt \textit{Franchisen}
	\item in der Rückversicherung: Selbstbehalt beim Erstversicherer, genannt \textit{Prioritäten}
	\item risikopolitisch und nicht gewinnorientierte Vorgehensweise
	\item für den Erstversicherer:
		\begin{itemize}
			\item Verringerung des versicherungstechnischen Risikos
			\item Erhöhung Zeichnungskapazität
			\item Solvenzverbesserung
			\item Kapitalkostenreduktion
		\end{itemize}
	\item für den Rückversicherer:
		\begin{itemize}
			\item Existenzgrundlage
			\item bessere Diversifikation der Risiken als beim Erstversicherer
		\end{itemize}
\end{itemize}

\subsubsection{Begrifflichkeiten Rückversicherung}
\begin{itemize}
	\item aktive RV: Angebot von Rückversicherungskaapzitäten
	\item passive RV: Nachfrage nach RV-Schutz durch Erstversicherer
	\item Retrozession: Weitergabe in Rückdeckung genommener Risiken eines RV an anderen RV
	\item obligatorische RV: Verpflichtung des Erstversicherers zur Übertragung aller vertraglich definierten Risiken ohne Ablehnungsrecht des RV
	\item fakultative RV: individuelle Abgabe und Annahme von Risiken auf einzelvertraglicher Basis
	\item Originalbasis: RV erhält anteilig Prämie und muss Deckungskapital bilden
	\item Risikobasis: RV erhält Risikobeitrag und bildet kein Deckungskapital
\end{itemize}

\subsubsection{Proportionale Risikoteilung}
\begin{itemize}
	\item proportionale Aufteilung der Schäden in festem Verhältnis zwischen Vertragspartnern
	\item Proportionen vorab fest und unabhängig von Schadenhöhen
	\item einfache Struktur, geringe Flexibilität
	\item bei RV Schicksalsteilung: Übernahme von Teilen des Erstversicherungsrisikos, aber nicht kaufmännischen oder unternehmerischen Risikos des Erstversicherers
	\item wichtigste Formen der proportionalen RV: Quotenrückversicherung, Summenexedentenrückversicherung
		\begin{itemize}
			\item QRV: feste Quotenabgabe $q$, Selbstbehalt $ \underbar{$S^{ind}$} = (1-q) \cdot S^{ind}$
			\item SERV: Festlegung eines Maximums $v_0$ als maximaler Selbstbehalt des Erstversicherers bei jedem einzelnen Risiko und vertragsindividuelle Quote $q_i = \frac{max\{v_i-v_0, \ 0\}}{v_i}$ in Abhängigkeit der jeweiligen Versicherungssumme
		\end{itemize}
	\item SERV dient der Homogenisierung des Portfolios und der Reduktion von Spitzenrisiken
	\item Üblicherweise Haftungsbegrenzung für RV i.H.v. Vielfachem $m$ von $v_0$. Mehrere aneinandergereiht, s.d. man Layering erhält.
\end{itemize}

\subsubsection{nicht-proportionale Risikoteilung}
\begin{itemize}
	\item alle, die keine proportionale Aufteilung vorsehen
	\item komplizierte Strukturen
	\item schwierige quantitative Analysierbarkeit
	\item gut zur Erreichung gezielter Effekte
	\item dominierend: Abzugsfranchise
		\begin{itemize}
			\item Franchisegrenze absoluter Höhe $a$ zwischen VN und VU
			\item Schaden in Höhe $X \ \Rightarrow \underbar{X}=min\{X,a\}$, d.h. VU übernimmt Teil, der $a$ übersteigt
			\item Jahresfranchise: analog mit Franchisegrenze und Jahresgesamtschaden
			\item Integralfranchise: wenn Grenze überschritten, übernimmt VU Schaden komplett
			\item Zeitfranchise: VN trägt jeden Schaden bis zum Ablauf der Frist selbst
		\end{itemize}
	\item wichtigste Formen: 
		\begin{itemize}
			\item Schadenexzedentenrückversicherung: Priorität $a$ und Limit $l$, Übernahme des Teils, der $a$ übersteigt und unter $l$ liegt. Wirkt pro Risiko, eignet sich für LCs. Limitierte Layer als Differenz zweier unlimitierter Layer darstellbar. 
			\item Kumulschadenexzedentenrückversicherung: Priorität $a^*$, Limit $l^*$ pro Kumulereignis (Anwendung auf Gesamtschäden von Kumulereignissen). Übernahme des die Prio übersteigenden Teils.
			\item Jahresüberschadenexzedentenrückversicherung: Anwendung auf Jahresgesamtschaden, ebenfalls Priorität und Limit
		\end{itemize}	
\end{itemize}

\subsubsection{Entschädigung}
\begin{itemize}
	\item Entschädigung $Z=g(X)$, $X \hat{=}$ Finanzaufwand, $g$ monoton wachsend, $g(x) \leq x$
	\item Selbstbehalt des VN $X-Z=X-g(X)$
	\item proportionale Selbstbeteiligung: $Z=g_1(X) \coloneqq q \cdot X, \ q \in (0,1)$
	\item Abzugsfranchise: $Z=g_2(X) \coloneqq (X-a)^+, \ a>0$
	\item Haftungsbegrenzung: $Z=g_3(X) \coloneqq min(X;I), \ I>0$
	\item allgemeine Darstellung der Entschädigung:
		\begin{equation}
			Z=q \cdot min\{(X-a)^+,I\}= \begin{cases}
				0 & X \leq a \\
				q \cdot (X-a) & a<X \leq a+I \\
				q \cdot I & a+i < X \\
			\end{cases}
		\end{equation}
	\item Prämie basiert auf Erwartungswert der Entschädigung $E(Z) \coloneqq E[q \cdot min \{(X-a)^+,I\}]= q \cdot \int_a^{a+I}(1-F(x))dx$, wobei $F(x)$ die Verteilungsfunktion der Finanzaufwände ist.
\end{itemize}



\chapter{Schadenversicheungsmathematik}

\subsubsection{Notation}
\begin{itemize}
\item $n$: Anzahl der Verträge
\item $\alpha_i$: Jahreseinheit des $i$-ten Vertrages, $i=1,...,n$
\item $v_i$: Versicherungssumme des $i$-ten Vertages
\item $b_i$: Jahresbeitrag des $i$-ten Vertrages
\item $N$: (zufällige )Anzahl Schäden
\item $X_j$: (zufällige) Höhe des $j$-ten Einzelschadens, $j=1,...,N$
\item $r$: Anzahl der Tarifmerkmale
\item $M_k$: $k$-tes Tarifmerkmal, $k=1,...,r$
\item $n_k$: Anzahl der verschiedenen Ausprägungen des $k$-ten Merkmals
\item $a_{j,k}$: $j$-te Ausprägung des $k$-Tarifmerkmals, $j=1,..., n_k$
\end{itemize}

\subsubsection{Schadenkennzahlen}
\begin{itemize}
\item Schadendaten: Zeitpunkt, Art und Ursache, Sachlicher Bezug, Ort, Entschädigung
\item Bestandsdaten: Versicherungssumme, persönliche Daten der VN, ...
\item Exposure: Das Risiko eines Vertrags oder eines Bestandes\\
Exposuremaß: versicherungstechnische Risiko bzw Schadenbedarf eines Bestandes (Bsp: Jahreseinheiten, Anzahl Risiken, Summe Beiträge, ...)
\item Anzahl Jahreseinheiten bzw. durchschnittliche Anzahl der Verträge: $n_0 := \sum_{i=1}^{n} \alpha_i$
\item Schadenhäufigkeit / Frequenz: $H:= \frac{\text{Anzahl Schäden}}{\text{Anzahl Jahreseinheiten}} = \frac{N}{n_0}$
\item Schadendurchschnitt: $D:= \frac{\text{Gesamtschaden}}{\text{Anzahl Schäden}} := \frac{\sum_{j=1}^N X_j}{N} = \frac{S}{N}$
\item Schadenbedarf: $SB:= \frac{\text{Gesamtschaden}}{\text{Anzahl Jahreseinheiten}} = \frac{S}{n_0} = H\cdot D = \text{Schadenhäufigkeit} \cdot \text{Schadendurchschnitt}$
\item Summe verdiente Beiträge: $b:= \sum_{i=1}^n \alpha_i \cdot b_i$
\item Schadenquote: $SQ:=\frac{\text{Gesamtschaden}}{\text{Summe verd. Beiträge}} = \frac{S}{b}$
\item durchschnittliche kumulierte Versicherungssumme: $v:= \sum_{i=1}^n \alpha_i \cdot v_i$
\item Schadensatz: $SS:=\frac{\text{Gesamtschaden}}{\text{durchschn. kumulierte V-Summe}} = \frac{S}{v}$
\item durchschnittliche Versicherungssumme: $v_0:= \frac{\text{durchschn. kumulierte V-Summe}}{\text{Anzahl Jahreseinheiten}} = \frac{v}{n_0}$
\item Schadengrad: $SG:=\frac{\text{Schadendurchschnitt}}{\text{durchschn. V-Summe}} = \frac{D}{v_0}$
\end{itemize}

\subsubsection{Grundlagen der Tarifierung}
\begin{itemize}
\item Jedem Risiko wird im Rahmen der Tarifierung zunächst der Vektor der Ausprägung $(a_{i_1,1}, a_{i_2, 2}, ..., a_{i_r, r})$ der ausgewählten Tarifmerkmale $M_1,...,M_r$ zugeordnet. Dieser Vektor legt dann genau ein der $t:= \prod_{k=1}^r n_k=$Anzahl der Tarifzellen verschiedenen Tarifzellen eindeutig fest.
\item Für jedes $r$-Tupel $(i_1, ..., i_r)$ und damir für jede Tarifzelle ist die Nettorisikoprämie $b_{i_1, ..., i_r}$ zu bestimmen
\item Tarifmodelle verwenden Schadenbedarfe und für jedes Merkmal $M_k$, $k=1,...,r$ und jede Ausprägung $a_{j,k}$, $j=1,...,n_k$ dieses $k$-ten Merkmals einen der insgesamt $n:= \sum_{k=1}^r n_k$ sogenannten Marginalparameter
\item Dieses Marginalfaktoren bzw. summanden repräsentieren die verschiedenen Ausprägungen der Merkmale und quantifizieren den mittleren Einfluss der Ausprägung auf die Schadenaufwendungen und sind zu schätzen
\item $u_{k,j}:=$ Marginalfaktor bzw. summand der $j$-ten Ausprägung des $k$-ten Merkmals
\item Multiplikatives Modell: $b_{i_1, ..., i_r}:= sb\cdot \sum_{k=1}^r u_{k,i_k}$
\item Additives Modell: $b_{i_1, ..., i_r} := sb + \sum_{k=1}^r u_{k, i_k}$
\end{itemize}

\subsubsection{Tarifierungsverfahren}
Alles multiplikative Modelle: \\
Vereinfachte Bezeichnungen: $A:=M_1$, $B:=M_2$ Merkmale mit $p:=n_1$, $q:=n_2$ Merkmalsausprägungen. Es gilt: $t=p\cdot q$ und $n=p+q$. Außerdem: $x_i:=u_{1,i}, i=1, ...,p$ und $y_j:=u_{2,j}, j=1,...,q$. Zusätzlich $s_{i,j}$ Gesamtschaden in Tarifzelle $(i,j)$, $v_{i,j}$ Volumenmaß und $sb_{i,j}=\frac{s_{i,j}}{v_{i,j}}$ Schadenbedarf. \\
Es ergeben sich die Marginaldurchschnitt: $sb_{i,\circ}:=\frac{s_{i \bullet}}{v_{i\bullet}}$ und $sb_{\circ, j} := \frac{s_{\bullet j}}{v_{\bullet j}}$ und der Schadenbedarf $sb:=\frac{s_{\bullet \bullet}}{v_{\bullet \bullet}}$ \\
Tarifierungsverfahren mit Marginaldurchschnitten:
\begin{itemize}
\item heuristischer Ansatz: Marginalfaktoren als normierte Marginaldurchschnitte der Risiken mit den jeweiligen Merkmalsausprägungen definieren:\\
$x_i^{MD} := \frac{sb_{i\circ}}{sb}$ \\
$y_j^{MD} := \frac{sb_{\circ j}}{sb}$ 
\item Dann folgt: $b_{i,j}^{MD} := sb \cdot x_i^{MD} \cdot y_j^{MD}$
\item Verfahren liefert meist wenig zufriedenstellende Ergebnisse $\rightarrow$ Weiterentwicklung erforderlich
\end{itemize}
Tarifierungsverfahren von Bailey $\&$ Simon:
\begin{itemize}
\item Ansatz orientiert sich an der Abstandsfunktion des $\chi^2$-Tests
\item Versucht die Marginalfaktoren $x_i$, $y_j$ so zu wählen, dass die Summe der (gewichteten) quadratischen Abstände zwischen den beobachteten Gesamtschäden $s_{ij}$ und den kumulierten Nettorisikoprämien $v_{i,j}\cdot b_{i,j} = v_{i,j}\cdot sb \cdot x_i \cdot y_j$ über alle Zellen minimiert wird.
\item Lösungen ergeben sich nach Nullsetzen der partiellen Ableitungen durch "p+q nichtlinearen Bestimmungsgleichungen" (nicht explizit lösbar, aber lösbar mit Fixpunktiteration)
\item Mit den Grenzwerten der Iteration werden die Marginalfaktoren $x_i^{BS}$ und $y_j^{BS}$ gegeben (nicht eindeutig bestimmt)
\item Es folgt: $b_{i,j}^{BS} := sb \cdot x_i^{BS} \cdot y_j^{BS}$
\item Verfahren reagiert empfindlich auf Ausreißer und überschätzt beobachteten Gesamtschaden
\end{itemize}
Marginalsummenverfahren
\begin{itemize}
\item Ansatz: Für jede Ausprägung eines der beiden Merkmale sollen die kumulierten Nettorisikoprämien mit den kumulierten Gesamtschäden übereinstimmen.
\item Grundlage zur Bestimmung der Marginalfaktoren sind Marginalsummengleichungen (\textbf{AUFSCHREIBEN??})
\item Marginalsummengleichungen ergeben nichtlineares Gleichungssystem mit $p+q$ Gleichungen und Unbekannten
\item Lösen ebenfalls mit Fixpunktiteration. 
\item Bei Konvergenz ergeben sich die Grenzwerte als Marginalfaktoren $x_i^{MS}$ und $y_j^{MS}$
\end{itemize}

\subsubsection{Auswahl der Tarifmerkmale}
\begin{figure}[ht]
	\centering
	\includegraphics[width = .8\textwidth]{Bilder/Tarifierung.png}
\end{figure}
1. Ermittlung von Risikomerkmalen
\begin{itemize}
\item Finden von statistisch signifikanten Merkmalen aus Daten zweierlei Arten: Schadendaten und Bestandsdaten
\item Auswahl potenzieller Risikomerkmale permanent prüfen
\item Besondere Beachtung der Großschäden durch "Kupierung": Schäden werden an sog. Kuperungsgrenzen abgeschnitten. Pro Schaden stellt die Grenze die max. Höhe dar, mit denen Schadenaufwendungen in den weiteren Analysen berücksichtigt werden
\end{itemize}
2. Auswahlkriterien einzelner Tarifmerkmale
\begin{itemize}
\item Wichtigste Kriterien: Signifikanz, möglichst unabhängig, Zulässigkeit, Messbarkeit, Anzahl der Tarifmerkmale, Stabilität/Robustheit, Bezug zum Risiko, Imageaspekte
\end{itemize}
3. Auswahlmethoden der Gesamtheit der Tarifmerkmale
\begin{itemize}
\item Einerseits: Gute Erklärung des Schadenaufkommens, andererseits: stochastische Unabhägigkeit oder Analysierung ihrer Wechselwirkung
\item Für die Tarifmerkmale gemeinsame Verteilung: $P^{X_1,...,X_r}$
\item Bei stochastischer Abhängigkeit steigt der Grad der Komplexität um $P$ zu berechnen deutlich an
\item Moderne Ansätze: Einsatz von Copulas für die Abhängigkeiten
\item Ansatz 1: Multiple Regressionsanalyse (\textbf{GENAUER? Kap 4})
\item Ansatz 2: Verfahren der schrittweisen Auswahl (\textbf{GENAUER?})
\end{itemize}


\subsubsection{Abwicklungsmuster}




\subsubsection{Basisverfahren der Schadenreservierung}

Chain-Ladder-Verfahren
\begin{itemize}
\item Chain-Ladder Faktoren:
\begin{equation}
\hat{\varphi}_k^{CL} := \frac{\sum_{j=0}^{n-k} S_{j,k}}{\sum_{j=0}^{n-k} S_{j,k-1}}= \sum_{j=0}^{n-k} \frac{S_{j,k-1}}{\sum_{h=0}^{n-k} S_{h,k-1}} \cdot \frac{S_{j,k}}{S_{j,k-1}}, k=1,...,n
\end{equation}
\item Alle beobachtbaren Schadenstände werden verwendet (sonst nichts)
\item Die aktuellen Schadenstände $S_{i,n-i}$, $i=0,...,n$, aus der Diagonale des Abwicklungsdreiecks mit Hilde der Chain-Ladder-Faktoren sukzessive auf das Niveau der späteren Abwicklungsjahre hochgerechnet.
\item Ergebnisse werden als Chain-Ladder-Prädiktoren bezeichnet:
\begin{equation}
\hat{S}_{i,k}^{CL}:= \hat{\varphi}_k^{CL} \cdot \hat{S}_{i,k-1}^{CL}
\end{equation}
\item Reserven für Anfalljahr $i$ ergeben sich aus der Differenz der Prädiktoren für den erwarteten Entschadenstand und dem aktuellen Schadenstand:
\begin{equation}
\hat{R}_i^{CL}:= \hat{S}_{i,n}^{CL} - S_{i,n-i}
\end{equation}
\end{itemize}
Loss-Development-Verfahren
\begin{itemize}
\item Stellt ein Abwicklungsmuster für Quoten und dass für diese Quoten $\gamma_0, ...,\gamma_n$ a-priori-Schätzer $\hat{\gamma}_0, ..., \hat{\gamma}_n$ unterliegen 
\item Aktuelle Schadenstände $S_{i, n-i}$ per Division durch Schätzer auf Niveau des letzten Abwicklungsjahr hochgerechnet und mit Faktor $\hat{\gamma}_k$ auf Niveau des $k$-ten Abwicklungsjahrs zurück skaliert:
\begin{equation}
\hat{S}_{i,k}^{LD} := \hat{Y}_k \cdot \frac{S_{i,n-i}}{\hat{Y}_{n-i}}
\end{equation}
\end{itemize}
Bornhuetter-Ferguson-Verfahren
\begin{itemize}
\item Ähnlich wie LD-Verfahren, aber zusätzlich zu a-priori Schätzer $\hat{\gamma}_0, ..., \hat{\gamma}_n$ für Quoten noch a-priori-Schätzer $\hat{\alpha}_0, ..., \hat{\alpha}_n$ für erwartete Schadenstände ($\alpha_i := E[S_{i.n}]$)
\item aktuelle Schadenstände mit Hilfe der a-priori Schätzer linear fortgeschrieben.
\begin{equation}
\hat{S}_{i,k}^{BF} := S_{i,n-i}+(\hat{\gamma}_k-\hat{\gamma}_{n-i})\cdot \hat{\alpha_i}
\end{equation}
\item Durch Differenzbildung werden die BF-Prädiktoren für Zuwächste errechnet:
\begin{equation}
\hat{Z}_{i,k}^{BF} := \hat{S}_{i,k}^{BF} - \hat{S}_{i,k-1}^{BF} = (\hat{\gamma}_k - \hat{\gamma}_{k-1}) \cdot \hat{\alpha}_i
\end{equation}
\item iteriertes Bornhuetter-Ferguson-Verfahren: Nach der ersten Anwendung werden die Prädiktoren $\hat{S}_{i,n}^{BF}$ anstatt der $\hat{\alpha}_i$ verwendet (Einfluss wird so reduziert). Verfahren wird so lange wiederholt bis sich Grenzwerte einstellen als Prädiktoren
\end{itemize}
Additives Verfahren/ Incremental-Loss-Ratio-Verfahren
\begin{itemize}
\item Basiert auf Annahme, dass es Volumenmaße $\pi_0, ..., \pi_n$ für die Anfalljahre $i=0,...,n$ gibt.
\item Außerdem Abwicklungsmuster für die erwarteten (Schaden-)Quoten-Zuwächse $\frac{E[Z_{i,k}]}{\pi_i}$, sodass es Parameter $\zeta_0,...,\zeta_n$ gibt mit $\zeta_k=\frac{E[Z_{i,k}]}{\pi_i}$
\item Schätzer für relative Zuwächste werden additive Schadenquotenzuwächse verwendet:
\begin{equation}
\zeta_k^{AD} := \frac{\sum_{j=0}^{n-k} Z_{j,k}}{\sum_{j=0}^{n-k} \pi_j} = \sum_{j=0}^{n-k} \frac{\pi_j}{\sum_{h=0}^{n-k} \pi_h} \cdot \frac{Z_{j,k}}{\pi_j}
\end{equation}
\item Schätzer setzen für den Zuwachs des $k$-ten Abwicklungsjahr die Summe sämtlicher Zuwächse im $k$-Abwicklungsjahr zur Summe der zugehörigen Volumenmaße
\item Mit diesen Schätzern erhält man Prädiktoren für Zuwächse und Schadenstände:
\begin{align}
\hat{Z}_{i,k} := \pi_i \cdot \hat{\zeta}_k^{AD} \\
\hat{S}_{i,k}^{AD} := S_{i,n-i} + \sum_{j=n-i+1}^k \hat{Z}_{i,j}^{AD}
\end{align}
\end{itemize}
Cape-Cod-Verfahren
\begin{itemize}
\item Wie Additives Verfahren Volumenmaße $\pi_0,...,\pi_n$. Zusätzlich a-priori-Schätzer $\hat{\gamma}_0,...,\hat{\gamma}_n$ für Quoten
\item Annahme: erwartete Schadenquoten unabhängig von Anfalljahr $i$, daher existiert Parameter $\kappa$: $\frac{E[S_{i,n}]}{\pi_i}=\kappa$
\item $\kappa$ wird geschätzt:
\begin{equation}
\hat{\kappa}^{CC} = \frac{\sum_{j=0}^n S_{j,n-j}}{\sum_{j=0}^n \hat{\gamma}_{n-j} \cdot \pi_j} = \sum_{j=0}^n \frac{\hat{\gamma}_{n-j} \cdot \pi_j}{\sum_{h=0}^n \hat{\gamma}_{n-h} \cdot \pi_h} \cdot \frac{S_{j,n-j}}{\hat{\gamma}_{n-j}\cdot \pi_j}
\end{equation}
\item Cape-Code-Prädiktoren: 
\begin{equation}
\hat{S}_{i,k}^{CC} := S_{i,n-i} + (\hat{\gamma}_k - \hat{\gamma}_{n-i}) \cdot \pi_i \cdot \hat{\kappa}^{CC}
\end{equation}
\end{itemize}
Alle Verfahren sind Spezialfälle von Bornhuetter-Ferguson-Verfahrens. Unterschieden nur durch die Schätzung der Parameter.
\begin{figure}[ht]
	\centering
	\includegraphics[width=.8\textwidth]{Bilder/Abwicklungsverfahren.png}
\end{figure}
\begin{figure}[ht]
	\centering
	\includegraphics[width=\textwidth]{Bilder/Abwicklungsverfahren_PundC.png}
\end{figure}

\subsubsection{Erweiterung der Basisverfahren}
1. Ausreißereffekte \\
Besser, wenn a-priori-Schätzer vorhanden. Daher Chain-Ladder sehr anfällig und Cape-Cod am besten

2. Inflation\\
Separationsverfahren bietet Möglichkeit, Kalendereffekte zu berücksichtigen und eine Art Inflationsvereinigung vorzunehmen. Kurze Zusammenfassung: Es gibt Parameter für Abwicklungsjahre und für Kalenderjahre und für die Zuwächse gilt: $E[Z_{i,k}]=v_i\cdot \gamma_{i+k}\cdot \vartheta_k$. Schätzung der Parameter durch Marginalsummengleichungen. Durch bspw. Extrapolation Schätzung der Inflationparameter der nächsten Jahre. Am Ende: $\hat{Z}_{i,k}= v_i \cdot \hat{\gamma}_{i+k} \cdot \hat{\vartheta}_k$ 

3. Nachlauf\\
Annahme: Schäden innerhalb von $n+1$ Jahren vollständig abgewickelt. In Ausnahmefällen noch nach dem $n$-ten Jahr Änderungen in Schadenständen -> "Nachlauf" \\











\chapter{Personenversicherungsmathematik}

\begin{figure}[ht]
	\centering
	\includegraphics[width=.8\textwidth]{Bilder/Korbinian.png}
\end{figure}

\subsubsection{Modell}
\begin{itemize}
	\item h Ereignisse bzgl. einer Person, das zuerst eintretende Ereignis führt zum Ausscheiden aus Gesamtheit
	\item $h$: Anzahl der Ausscheideursachen, $T_i$ Zeitpunkt Eintritt des Ereignisses $i$, $X_i$ Alter bei Eintritt des Ereignisses $i$
	\item $X_i = T_i-t^*$, $t^* \hat{=}$ Geburtszeitpunkt, $G \coloneqq \lfloor t^* \rfloor$ das Geburtsjahr
	\item $X\coloneqq min\{X_i\}$ Ausscheidealter, $U\coloneqq min\{i \in \{1,...,h\}:X_i=X\}$: Ausscheideursache
	\item $\tensor[_{1}]{q}{}_x^{(i)}(G)$: Wahrscheinlichkeit einer $x$-jährigen Person der HGSH mit Geburtsjahr $G$ innerhalb des Zeitintervalls $(x,x+1)$ mit der Ursache $i$ auszuscheiden. Verbleibewahrscheinlichkeit $p_x \coloneqq 1-q_x$
	\item Allgemein: 
	\begin{align}
		\tensor[_{s}]{q}{}_x(G)\coloneqq \mathbb{P}[X \leq x+s | X>x] \\
		\tensor[_{s}]{p}{}_x(G) \coloneqq 1-\tensor[_{s}]{q}{}_x(G) = \mathbb{P}[X>x+s|X>x]
	\end{align}
	\item mehrjährige Verbleibewahrscheinlichkeit (noch mind. $k$ Jahre in HGSH verbleiben): $\tensor[_{k}]{p}{}_x = \prod_{j=0}^{k-1}p_{x+j}$
\end{itemize}













\end{document}